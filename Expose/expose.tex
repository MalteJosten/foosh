\documentclass[a4paper,11pt]{article}

\usepackage[ngerman]{babel}         % Neue deutsche Rechtschreibung
\usepackage[T1]{fontenc}            % Bessere Schriftdarstellung
\usepackage{lmodern}                % Aktuelle Schrift

\usepackage[intlimits]{amsmath}     % Zusaetzliche Matheumgebungen
\usepackage{amssymb}                % Mathematische Symbole
\usepackage{graphicx}               % notwendig fuer \includegraphics
\usepackage{fancyhdr}               % Kopf- und Fusszeile
\usepackage{lastpage}               % erzeugt Referez zu der letzten Seite
\usepackage{moreverb}               % verbatimtab Umgeung
\usepackage{tikz}
\usepackage{xcolor}
\usetikzlibrary{decorations.pathmorphing,shapes,decorations.text,decorations,calc,positioning,automata,matrix,arrows}


% Seiteneinstelungen
\setlength\textwidth{165mm}           % Breite
\setlength\textheight{235mm}          % Hoehe
\setlength\headheight{41pt}           % Hoehe der Kopfzeile
\setlength\topmargin{-12mm}           % Abstand oben
\setlength\oddsidemargin{0mm}         % Linker Rand
\setlength\parindent{0pt}             % und ohne Einrueckung
\setlength\parskip{1.7\medskipamount} % Absaetze abgesetzt
\sloppy\pagestyle{fancy}

% Kopf- und Fusszeileeinstellungen
\renewcommand{\headrulewidth}{0.4pt} 	%obere Trennlinie
\fancyfoot[C]{Page:~\thepage~of~\pageref{LastPage}} %Seitennummer
\renewcommand{\footrulewidth}{0.4pt} 	%untere Trennlinie

\newcommand{\R}{\mathbb{R}}                 % reelle Zahlen
\newcommand{\N}{\mathbb{N}}                 % natuerliche Zahlen
\newcommand{\e}{\text{e}}                   % eulersche Zahl
\newcommand{\E}[1]{\cdot10^{#1}}            % x 10^{...}
\newcommand{\qed}{\hspace*{\fill}q.e.d.}    % Beweis fertig
\newcommand{\ON}[1]{{\cal O}(#1)}	          %O-Notation

\newcommand{\todo}[1]{\marginpar{\textcolor{red}{TODO: #1}}}

\fancyhead[R]{Malte Josten, 3066184}
\fancyhead[C]{\large{\bf{<Eins Cooler Titel>}}}
\fancyhead[L]{\textbf{Master Thesis - Expose}\\ Summer 2023}

\usepackage{listings}

% Unteraufgaben (mit Enumeration)
\def\labelenumi{(\arabic{enumi})}
\parindent0mm % keine Absatzeinrückung

\usepackage[utf8]{inputenc}

\usepackage{enumitem}
\setlist{nosep}

\usepackage{dirtree}

\begin{document}
\section{Background}
\begin{itemize}
    \item[*] What's the goal?
    \item[*] How do I want to accomplish it?
    \item[*] Why is it necessary to do it?
\end{itemize}

\section{What Do I Want To Do?}
\subsection{The General Idea}
\begin{itemize}
    \item[*] Framework Overview
    \item[*] Go through, what each part is and what its capabilities are.
    \item[*] Show what's scientific about it.
    \item[*] How to validate/evaluate?
\end{itemize}

\subsection{(Example) Structure}
\dirtree{%
    .1 .
    .2 Introduction.
    .3 Background (current SmartHome technologies etc.).
    .2 Previous/Related Work.
    .2 Implementation.
    .3 Available Data (small set of real world measurements).
    .3 Artificial Intelligence (\textit{$\leftarrow$ more specific}).
    .4 Network Model.
    .4 Validation.
    .3 Simulation.
    .4 Base Model.
    .4 How to simulate temperature, brightness, etc..
    .4 What is used for simulation.
    .5 User input.
    .5 Inferred data.
    .3 Symbolic Regression.
    .4 ?.
    .2 Evaluation.
    .3 Validation (with real world data).
    .3 Comparison between different implementations.
    .2 Conclusion and Future Work.
}

\section{Related Work}
\# TODO

\section{Time Management}
\begin{itemize}
    \item[*] Time table
\end{itemize}


\end{document}
